% This work is licensed under the Creative Commons Attribution-NonCommercial 4.0 International License.
% To view a copy of this license, visit http://creativecommons.org/licenses/by-nc/4.0/
% or send a letter to Creative Commons, PO Box 1866, Mountain View, CA 94042, USA.

% !TEX TS-program = xelatex

\documentclass[../Main/chem371-notes.tex]{subfiles}

\setcounter{chapter}{2}
\begin{document}

\chapter{Hartree--Fock Theory}

\section{Hartree--Fock theory and the independent particle approximation}
The simplest quantum chemistry method is the Hartree--Fock approach.
This method relies on an independent particle approximation, where the wave function is expressed as a product of single electron orbitals.
For a system with $N$ electrons this approximation is given by
\begin{equation}
\Psi(x_1, x_2, \ldots, x_n) \approx \psi_1(x_1) \psi_2(x_2) \cdots \psi_N(x_N)
\end{equation}
This approximation is convenient because it breaks down the problem of solving the Schr\"{o}dinger for a wave function of $3N$ coordinates into a simpler problem where we solve $N$ times for wave functions of 3 coordinates.\mtodo{Add illustration}
This simplification makes the Hartree--Fock very affordable.

This wave function, however, does not represent the true properties of electrons.
We know that electrons have to satisfy the Pauli principle, which mathematically is equivalent to requiring that $\Psi(x_1, x_2, \ldots, x_n)$ changes sign if we exchange the label of two particles.
To illustrate this point, consider two electrons. The wave function $\Psi(x_1, x_2)$ depends on the position and spin of each electron ($x_1$, $x_2$), and it must change sign when we replace $x_2$ with $x_1$ and vice versa, which means
\begin{iequation}
\Psi(x_1, x_2) = -\Psi(x_2, x_1)
\end{iequation}
It is easy to write down a wave function built from orbitals that satisfies this condition, it just requires making the product antisymmetric with respect to the coordinates.
Such a wave function is called a \emph{Slater determinant} and for the two electron case is it written as
\begin{equation}
\Psi_{\mathrm{SD}}(x_1,x_2) = \frac{1}{\sqrt{2}} 
\begin{vmatrix}
\psi_1(x_1) & \psi_1(x_2) \\
\psi_2(x_1) & \psi_2(x_2)
\end{vmatrix}
= \frac{1}{\sqrt{2}} \left[ \psi_1(x_1)\psi_2(x_2) - \psi_1(x_2)\psi_2(x_1) \right]
\end{equation}

\section{The Hartree--Fock equation}
The main goal of the Hartree--Fock method is to find the orbitals that give the ``best'' possible wave function.
Here by ``best'' we mean precisely the one that minimizes the expectation value of the energy, which is given by the integral
\begin{equation}
E_\mathrm{SD} = \int \Psi_{\mathrm{SD}}^* \hat{H}\Psi_{\mathrm{SD}}
\end{equation}
The \emph{variational} theorem guarantees that the energy of an approximate wave function $\tilde{\Psi}$ is always greater than the exact energy, $\tilde{E} \geq E_\mathrm{exact}$, and the closer we get to the exact energy $E_\mathrm{exact}$, the better our wave function $\tilde{\Psi}$ approximate the exact solution $\Psi_\mathrm{exact}$.

The orbitals that minimize the energy satisfy the \emph{Hartree--Fock equations}
\begin{equation}
\hat{f} \psi_i(x) = \epsilon_i  \psi_i(x), \quad \text{ for all } \psi_i(x)
\end{equation}
where $\hat{f}$ is the \emph{Fock operator}.
This is an eigenvalue equation like the Schr\"{o}dinger equation, but it is different because it is valid only in the Hartree--Fock approximation and it is simpler because it only involves functions of 3 coordinates.
The quantity $\epsilon_i$ is called the \emph{orbital energy} and it is the eigenvalue corresponding to the orbital $\psi_i$.

The physical interpretation of the Hartree--Fock equation is that we can approximate the interaction (repulsion) of an electron in orbital $\psi_i$ with all the remaining $N-1$ electrons via an average potential. This potential is included in the Fock operator ($\hat{f}$).
Since the average repulsive potential depends on how the electrons are distributed, the Hartree--Fock equation have to be solved via an iterative \emph{self-consistent-field (SCF) procedure}.\mtodo{Add example of SCF and illustration}
This procedure consists of the following steps:
\begin{enumerate}
\item Forming an initial guess. At beginning of an SCF procedure we need to start from orbitals that are close to the optimal ones.
Common ways to guess the orbitals include neglecting electron repulsion or forming the average potential from a superposition of atomic densities.
\item Updating the Fock matrix. Using the current set of orbitals, the average potential and the Fock operator are built. 
\item Determining the orbitals and orbital energies. Using the current Fock matrix, the Hartree--Fock equations are solved to obtain one set of orbitals and orbital energies and an updated value for the total energy.
\item Convergence check. The program checks if the change in energy and orbitals with respect to the previous iteration is less than a predefined convergence threshold. If the computation has converged, we stop. Otherwise, we go back to step 2. and use the new set of orbital to compute a new Fock operator.
\end{enumerate}


\section{Interpretation of the orbital energies}
It is important to understand the precise meaning of the orbital energies $\epsilon_i$ because it is slightly different than our intuition as chemists.
\emph{Koopman's theorem} shows that the orbital energies are related to the energy necessary to remove or add electrons to an orbital.
More precisely, for orbitals that are occupied by electrons, Koopman's theorem says that $-\epsilon_i$ correspond to the energy necessary to remove one electron (ionize) from the system
\begin{equation}
\ce{M} \rightarrow \ce{M+} + e^{-} 
\end{equation}
Koopman's theorem can be stated as
\begin{iequation}
\text{Ionization potential for electron in }\psi_i = \mathrm{IP}_i = E^{N-1} - E^{N} = -\epsilon_i.
\end{iequation}
If an occupied has a negative orbital energy, then we have to put energy into a molecule to remove that electron (the IP is positive).

Similarly, for orbitals that are not occupied, the orbital energy $\epsilon_a$ corresponds to the energy released when an electron is added to form an anion (if the starting molecule is neutral)
\begin{equation}
\ce{M} + e^{-} \rightarrow \ce{M-}  
\end{equation}
In this case, Koopman's theorem helps us quantify the electron affinity
\begin{iequation}
\text{Electron affinity for electron in }\psi_a = \mathrm{EA}_a = E^{N} - E^{N+1} = -\epsilon_a.
\end{iequation}
Therefore, if an unoccupied orbital has a negative energy (positive EA), a molecule will attach an electron and form a stable negative ion.
If instead $\epsilon_a$ is positive, a molecule will not accept an electron, and its anion will be unstable with respect to self ionization (the anion will spontaneously loose an electron in a finite amount of time).

Koopman's theorem assumes that during the electron ionization/attachment the orbitals do not change, which means that it ignores relaxation effects to the addition or removal of electrons.
Koopman's theorem also neglects electron correlation, so the IP and EA will deviate from experiment.
This means that in practice Koopman's theorem is not a very accurate way to compute the IP and EA of molecules.
Nevertheless, it gives physical meaning to the orbital energies and it can be used to qualitatively estimate the IP and EA of molecules.

\section{Interpretation of the orbitals}



\end{document}